\documentclass{article} % For LaTeX2e
\usepackage{nips14submit_e,times}
\usepackage{hyperref}
\usepackage{url}
%\documentstyle[nips14submit_09,times,art10]{article} % For LaTeX 2.09


\title{Melody Extraction \\ Final Report: DSGA-1003 Machine Learning}



\author{
Justin Mao-Jones \\
Center for Data Science\\
New York University\\
\texttt{justinmaojones@nyu.edu} \\
\And
Junbo Zhao \\
Center for Data Science\\
New York University\\
\texttt{j.zhao@nyu.edu} \\
\AND
Rita Li \\
Center for Data Science\\
New York University\\
\texttt{ml4713@nyu.edu} \\
}

% The \author macro works with any number of authors. There are two commands
% used to separate the names and addresses of multiple authors: \And and \AND.
%
% Using \And between authors leaves it to \LaTeX{} to determine where to break
% the lines. Using \AND forces a linebreak at that point. So, if \LaTeX{}
% puts 3 of 4 authors names on the first line, and the last on the second
% line, try using \AND instead of \And before the third author name.

\newcommand{\fix}{\marginpar{FIX}}
\newcommand{\new}{\marginpar{NEW}}

%\nipsfinalcopy % Uncomment for camera-ready version

\begin{document}


\maketitle


\section{Introduction}

In the project, we aim to build and evaluate machine learning models for the task of extracting melody from digital music files.  Melody extraction is an active research area in the music information retrieval research community [1].  It has many applications, such as query by humming (e.g. hum a song into your phone and an app tells you what song it is), cover song identification, genre classification, and mood classification.

Generally speaking, melody is the predominant pitch in a piece of music that captures the essence of a song.  As a motivating example, the melody is the tune one might hum when asked "what does the song sound like?"  Unfortunately, there does not seem to be a precise definition of melody.  As a machine learning task, we require some sort of precise definition, and for this project we adopt the definition described by [1]:

\begin{quote}
Melody is the fundamental frequency \footnote{Any audio signal can be represented as a sum of a series of sinusoids.  The fundamental frequency of a signal defined as the lowest frequency in the series.  It can be derived through the Fourier Transform.} from musical content with a lead voice or instrument.  Melody extraction is the estimation of this melody from a single source.
\end{quote}

While this definition is still open to interpretation, it can be used by human experts to generate the melody of a piece of music.  Note that we constrain the definition of melody extraction to a single source, meaning that the melody is only coming from a single lead voice or instrument.  The motivation for this simplification is that it can make the task of melody extraction easier.

Melody extraction is a supervised learning task and requires a set of labeled data.  For this project, we used MedleyDB [2], a dataset of annotated, royalty-free multitrack recordings.

Many approaches to melody extraction have been attempted [1], including pure signal processing [3], dynamic programming [4], support vector machines [5], and hidden markov models [6] [8].  In this project, we utilized SVM, HMM, and deep neural network architectures.  Our best performing model was a recurrent neural network (RNN) with long short-term memory (LSTM) cells.

\subsection{Review of Existing Work}

MedleyDB was developed relatively recently, and so not much work has been published on it.  The only melody extraction result on MedleyDB that we are aware of is that of Melodia, which has continued to set the state-of-the-art on the MIREX09 dataset from 2011-2014.  MIREX09 is popular dataset for testing melody extraction algorithms, and so we do a quick review of those results here.

\section{Problem Definition}
As a machine learning task, melody extraction has two components:

\begin{enumerate}
\item Voicing detection (i.e. classification of whether or not the melody is present),
\item Melody pitch tracking.
\end{enumerate}

The first task derives from the fact that sometimes there is no melody.  The tasks can be approached in separate algorithms or in the same algorithm.

Melody pitch tracking is the task of identifying the predominant frequency during a \textit{frame} (a small time interval).  We are effectively constrained to the frame size used in the MedleyDB melody annotations, and thus our frames are roughly 4ms.

An important modeling question is whether to predict melody labels through regression or classification.  Regression can seem to be a natural fit, because frequency lies on a continuous spectrum.  However, human music tends to be composed on a discrete scale, i.e. musical notes.  Predictions could be refined by "rounding" the predictions to the nearest note.

As a classification task, we can model all possible notes, such as the 88 keys found on a standard piano.  Typically, most songs may not even cover this entire spectrum, and so we can reduce the scope of our classification to only those notes present in the data.

\subsection{Melody Specific Challenges}

There are two major difficulties with melody.  First, the melody is often mixed in with other musical signals.  For example, the melody might be produced by the singer, but there is also a guitar, a bass guitar, and drum beats playing at the same time.  

Second, it is likely that melody is context dependent.  For example, it might actually be impossible for any algorithm to detect discernable melody patterns in a 4 ms window without any additional information.  Intuitively, it would seem that one must listen to the music surrounding that specific time frame in order to determine the melody.  Thus, the features of a specific sample instance should include audio signal information from surrounding time frames.  The number of such features could be potentially large, possibly huge.  How large of a time window is needed to identify a melody?  An additional related question is whether or not future information is required to predict melodies.

\section{Data} \label{sec:data}

MedleyDB [2] consists of 122 multitracks, including stereo quality mixed audio, melody annotations, and stems \footnote{In a recording session, there is a separate microphone for each instrument (or sets of instruments), and thus there are separate recordings.  For example, the singer is recorded separately from the guitar.  A stem is one of these separate recordings.  When added together the mix will sound like a complete song.}.  The multitracks include songs from a variety of genres, including Singer/Songwriter, Classical, Rock, World/Folk, Fusion, Jazz, Pop, Musical Theatre, and Rap.

Only 108 out of the 122 songs contain melody annotations, and thus these are the songs we use for our experiments.

The audio files are provided in WAV format (44.1 kHz, 16 bit).  In other words, each audio file contains 44,100 digital audio samples per second.  Each audio file is accompanied by a single source melody annotation, provided in csv format.  A melody label is a number that represents the predominant frequency over a pre-defined window of time.  There are 256 melody labels per second.  Each melody label overlaps roughly 172 audio samples.

One of the benefits of MedleyDB is that it was carefully curated to provide a complete set of melody annotations combined with high-quality songs.  Thus, we are operating under the assumption that we do not have missing data and that the melody annotations perfectly overlap with the audio samples.  Unfortunately, it would not be practically feasible to check this assumption thoroughly.

A question we explore throughout the project is whether or not we have enough data.  On the one hand, 122 songs does not sound large.  On the other hand, 256 melody labels per second, at an average of 3 minutes per song, corresponds to over 5 million training samples.  So the question is whether or not there is enough variety in this dataset to generalize well to other datasets or even between the training, validation, and test splits.

\subsection{Data Pre-processing}

MedleyDB comes with an accompanying API for working with its data files, and so very little data pre-processing was required.  The majority of data processing work was in the form of feature generation, such as CQT transforms, and generating training, validation, and test splits.

\section{Evaluation Metrics For Model Performance}
Given that melody extraction consists of two tasks, it is natural to evaluate each task separately and together, thus yielding three different evaluation metrics.  Typically, researchers use accuracy to evaluate performance \footnote{http://www.music-ir.org/mirex/wiki/2014:Audio\_Melody\_Extraction}, and so we will do the same here by following the MIREX accuracy definitions, with one small modification:

\begin{enumerate}
\item Voicing detection TPR, i.e. probability that a frame that is actually voiced is predicted to be voiced,
\item Raw pitch accuracy - probability of a correct pitch value (within 1/4 tone\footnote{a tone is a step between two keys on a piano; e.g. G is a tone higher than F.}) given that the frame is voiced,
\item Overall accuracy = $\frac{TPC+TN}{TO}$


TP = total number of frames correctly predicted as voiced

TPC = total number of TP frames in which pitch was also correctly predicted

TN = total number of frames correctly predicted as unvoiced

TO = total number of predictions

\end{enumerate}

Our modification has to do with voicing detection.  Throughout the project, we take a general strategy of first implementing our methods to the voicing detection task, and then to overall melody extraction.  The reasoning is that we presumed the former to be a simpler task than pitch tracking. For the voicing detection tasks, TPR would not be enough to accurately represent the ability of a machine learning algorithm to predict the presence of melody.  Thus, we instead use voicing detection accuracy, which is the total number of correct predictions divided by the total number of predictions.

For overall melody extraction, we look at a combination of raw pitch accuracy and overall accuracy, as this seems to be the common approach in the MIR community.

\section{Training/Validation/Testing}

We split our training, validation, and testing sets on songs.  In other words, no songs will overlap between sets.  This intuitively makes sense.  In a real world application, a melody extraction procedure would be applied to songs it had never seen before. 

Given the limited number of songs in our dataset, a possible approach would be to use cross-validation.  However, running cross-validation on deep-learning architectures would be computationally expensive and not practical.  We also wanted to maintain the same training, validation, and test splits for all our machine learning approaches.  Thus we elected not to use cross-vadliation.

As mentioned previously, we use only the 108 songs that contain melody annotations.  Our splits consisted of 27 songs for the test set, 27 songs for validation, and the remaining 54 songs for training.  These were fixed at the beginning of the project.

\section{Feature Extraction}
Since Music Information Retrieval (MIR) is closely related to the Speech Recognition (SR) community, it naturally absorbs feature descriptors and 
machine learning techniques from the Speech community, which has a relatively larger amount of literature.
We briefly studied the marriage between the communities and settled a few methods that might be appropriate. 

One of the difficulties in both MIR and SR is that, in order to progress in model performance, task-specific handcrafted features are sometimes needed.  We acknowledge this, but have decided that we will start with off-the-shell features such as Mel-Frequency Cepstral Coefficients (MFCC) [1], Short-time Fourier Transform (STFT) and multi-resolution FFT (MRFFT) [8].

\begin{itemize}
\item \textbf{STFT}. STFT is a widely used signal preprocessing technique. In general, the signal is chunked into frames where STFT is applied to each chunked frame, with a window length typically assigned as 50 and 100ms. Frames can overlap in a sliding window fashion.

\item \textbf{MRFFT}. Resolution issues inherently arise with Fourier transform. MRFFT overcomes this by taking frequency spectrum out of multi-resolution windows.

\item \textbf{MFCC}. This has been the dominant feature descriptor in Speech community over the past 30 years. It is basically a linear cosine encoding of the log power spectrum on a mel-scale of frequency which biologically originated from human's ears. 

\item \textbf{Dictionary Learning}. Dictionary learning is an adaptive content-based feature self-learning approach. Its goal basically is to get local descriptions by learning a linear combination of a pre-defined dictionary. 
The weights of the “words” in the dictionary are the new representation of the local window. The process of getting the dictionary is unsupervised; K-means and K-SVD [9] are two common methods to obtain this dictionary.
\end{itemize}

In general, Fourier transform intuitively seems to be natural to the problem of melody extraction.  The human auditory system naturally perform a fourier analysis in the conversion of sound to neural impulses.  In addition, the fourier transform generates a time series of frequency vectors, and the melody traverses across these vectors through time.

\section{Baseline Models}

It is important to have a baseline performance to measure the performance of a model against.  We will use the following baseline models:

\begin{itemize}
\item Regression:
\subitem Pitch Tracking: Predict the average melody pitch
\item Classification:
\subitem Voicing Detection:  Predict the majority class (on/off)
\subitem Pitch Tracking: Predict the majority class
\end{itemize}



\subsubsection*{References}


\small{
[1] J. Salamon, E. Gomez, D. P. W. Ellis and G. Richard, "Melody Extraction from Polyphonic Music Signals: Approaches, Applications and Challenges", IEEE Signal Processing Magazine, 31(2):118-134, Mar. 2014.

[2] R. Bittner, J. Salamon, M. Tierney, M. Mauch, C. Cannam and J. P. Bello, "MedleyDB: A Multitrack Dataset for Annotation-Intensive MIR Research", in 15th International Society for Music Information Retrieval Conference, Taipei, Taiwan, Oct. 2014.

[3] J. Salamon and E. Gómez, "Melody Extraction from Polyphonic Music Signals using Pitch Contour Characteristics", IEEE Transactions on Audio, Speech and Language Processing, 20(6):1759-1770, Aug. 2012.

[4] V. Rao and P. Rao, “Vocal melody extraction in the presence of pitched accompaniment
in polyphonic music,” IEEE Trans. Audio, Speech, Lang. Processing, vol. 18,
no. 8, pp. 2145–2154, Nov. 2010. 

[5] G. Poliner and D. Ellis, “A classification approach to melody transcription,” in Proc. 6th Int. Conf. Music Information Retrieval, London, Sept. 2005, pp. 161–166. 

[6] M. Ryynänen and A. Klapuri, “Automatic transcription of melody, bass line, and chords in polyphonic music,” Comput. Music J., vol. 32, no. 3, pp. 72–86, 2008. 

[7] Masataka Goto, Hiroki Hashiguchi, Takuichi Nishimura, and Ryuichi Oka: RWC Music Database: Popular, Classical, and Jazz Music Databases, Proceedings of the 3rd International Conference on Music Information Retrieval (ISMIR 2002), pp.287-288, October 2002. 

[8] T.-C. Yeh, M.-J. Wu, J.-S. Jang, W.-L. Chang, and I.-B. Liao, “A hybrid approach to singing pitch extraction based on trend estimation and hidden Markov models,” in IEEE Int. Conf. Acoustics, Speech, and Signal Processing (ICASSP), Kyoto, Japan, Mar. 2012, pp. 457–460.

[9] Aharon, Michal and Elad, Michael and Bruckstein, Alfred, "K-SVD: An Algorithm for Designing Overcomplete Dictionaries for Sparse Representation," in IEEE Transactions on Signal Procesing 2006, Vol 54 No 11 pp. 4311.
}


\end{document}